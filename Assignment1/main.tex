\let\negmedspace\undefined
\let\negthickspace\undefined
\documentclass[journal,12pt,twocolumn]{IEEEtran}
\usepackage{gensymb}
\usepackage{amssymb}
\usepackage[cmex10]{amsmath}
\usepackage{amsthm}
\usepackage[export]{adjustbox}
\usepackage{bm}
\usepackage{longtable}
\usepackage{enumitem}
\usepackage{mathtools}
\usepackage{tikz}
\usepackage[breaklinks=true]{hyperref}
\usepackage{listings}
\usepackage{color}                                            %%
\usepackage{array}                                            %%
\usepackage{longtable}                                        %%
\usepackage{calc}                                             %%
\usepackage{multirow}                                         %%
\usepackage{hhline}                                           %%
\usepackage{ifthen}                                           %%
\usepackage{lscape}     
\usepackage{multicol}
\DeclareMathOperator*{\Res}{Res}
\renewcommand\thesection{\arabic{section}}
\renewcommand\thesubsection{\thesection.\arabic{subsection}}
\renewcommand\thesubsubsection{\thesubsection.\arabic{subsubsection}}
\renewcommand\thesectiondis{\arabic{section}}
\renewcommand\thesubsectiondis{\thesectiondis.\arabic{subsection}}
\renewcommand\thesubsubsectiondis{\thesubsectiondis.\arabic{subsubsection}}
\hyphenation{op-tical net-works semi-conduc-tor}
\def\inputGnumericTable{}                                 %%
\lstset{
	frame=single, 
	breaklines=true,
	columns=fullflexible
}

\usepackage{graphicx}

\begin{document}
	\title{\huge{Assignment 1}\\AI1110}
	\author{\Large{I SAI PRADEEP}\\AI21BTECH11013}
	\maketitle
	\begin{abstract}
		This document contains the solution to problem 3 (c) in 10th ICSE 2019 Paper
	\end{abstract}
	
	%Question
	\noindent \textbf{Question 3(c)} A solid metallic sphere of radius 6cm is melted and made into a solid cylinder of height 32cm. Find the
	\begin{enumerate}[label=(\roman*)]
		\item radius of the cylinder
		\item curved surface area of the cylinder
	\end{enumerate}	
	%Solution

	\textbf{Solution:} The various parameters involved in this question are listed in Table \eqref{table:table1}:
	\begin{table}[!ht]
		\input{tables/table.tex}
		\caption{}
		\label{table:table1}	
	\end{table}

\begin{figure}[h!]
	\centering
	\includegraphics[width=0.5\columnwidth]{fig_sphere.png}
	\caption{sphere of radius 6cm}
	\label{sphere}
\end{figure}

	The Volume of a sphere of radius R is:
	\begin{align} 
		4/3 \times \pi \times R^3
	\end{align}
	
	Let $V$ be the volume of the metallic sphere, of radius 6cm. 
	Hence,
	\begin{align} V&=4/3 \times \pi \times 6^3 cm^3 
		\label{eq:sphere}
	\end{align} 
	\begin{enumerate}[label=(\roman*)]
		\item When the sphere is melted, it's volume doesn't change
		Since the cylinder is obtained by melting the sphere, Hence the volume of the cylinder is same as that of the sphere.
		Let radius of the cylinder be r cm.
		The volume of this cylinder is given as: 
		\begin{align} 
		Volume &= \pi \times r^2 \times h cm^3
			\label{eq:cylinder}
		\end{align}
		where h=32cm
		Using Equations 
		\eqref{eq:sphere} and 
		\eqref{eq:cylinder} 
		,equating them,we get 
		\begin{align}
		4/3 \times \pi \times 6^3 cm^3&=\pi \times r^2 \times 32 cm^3
			\\
		\implies 4/3 \times 216&=r^2 \times 32
			\\
		\implies 288&= r^2 \times 32
			\\
		\implies r^2&= 288/32
			\\
		\implies r^2&=9
			\\
		\implies r&=3cm
		\end{align} 
	$\therefore$  radius of the  cylinder is 3 cm
		
		\begin{figure}[h!]
			\centering
			\includegraphics[width=0.5\columnwidth]{fig_cylinder.png}
			\caption{cylinder of radius 3cm and height 32cm}
			\label{cylinder}
		\end{figure}
		\item The curved surface area ( CSA ) of a cylinder in centimeter square units with radius r cm and height h cm is given as
		$ 2 \times \pi \times r \times h $
		\\
		Let $C$ be the CSA(curved surface area) of the cylinder.
		\begin{align}	
		\implies C&=2 \times \pi \times r \times h
		\end{align}
		
		Here for the cylinder r=3cm and h=32cm\\
		%	Substituting the values of r and h respectively as 3 cm and 32 cm,
		\begin{align}	
		\implies C&=2 \times \pi \times 3 \times 32
			\\
		\implies C&=192 \times \pi 
			\\
		%	Substituting the approximate value of pi in the Equation (3), we get
		\implies C&=603.185789 cm^2 
		\end{align}
	
	\end{enumerate}
		$\therefore$ The Curved Surface Area of cylinder is nearly 603 square centimeters.
	\ifthenelse{\isundefined{\languageshorthands}}{}{\languageshorthands{\languagename}}\gnumericTableEnds
\end{document}
