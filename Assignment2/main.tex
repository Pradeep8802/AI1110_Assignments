\documentclass[journal,12pt,twocolumn]{IEEEtran}
\usepackage{amsmath}
\providecommand{\pr}[1]{\ensuremath{\Pr\left(#1\right)}}
\providecommand{\cbrak}[1]{\ensuremath{\left\{#1\right\}}}
\newcommand*{\permcomb}[4][0mu]{{{}^{#3}\mkern#1#2_{#4}}}
\newcommand*{\comb}[1][-1mu]{\permcomb[#1]{C}}
\begin{document}
	\title{\huge{Assignment 2}\\AI1110}
	\author{\Large{I Sai Pradeep}\\AI21BTECH11013}
	\maketitle
	\begin{abstract}
		This document contains the solution to problem 1(ix) in 12th ICSE 2019 Paper
	\end{abstract}
	%Question
	\noindent \textbf{Question 1(ix):}
	Two balls are drawn from an urn containing 3 white,5 red and 2 black balls, one by one without replacement. What is the probability that at least one ball is red?\\	
	%Solution
	\textbf{Solution:} Let $X=\cbrak{0,1,2}$ be a random variable representing the colour of the ball, and let $Y=\cbrak{0,1}$ be a random variable representing the draw number. Let $P_7$ be the probability that atleast one of the balls drawn is red.
	See Tables 
	\eqref{table:prob1}
	and 
	\eqref{table:prob2} 
	\begin{table}[ht!]
		\begin{tabular}{|l|c|}
	\hline
	\textbf{Event} & \textbf{Description} \\
	\hline
	$X = 0$ &  colour of the ball drawn is Black \\
	\hline
	$X = 1$ &  colour of the ball drawn is White \\
	\hline
	$X = 2$ &  colour of the ball drawn is Red \\
	\hline
	$Y = 0$ &  The first draw of the balls \\
	\hline
	$Y = 1$ &  The second draw of the balls \\
	\hline
\end{tabular}

		\vspace*{5pt}
		\caption{}
		\label{table:prob1}
	\end{table}
	\begin{table}[ht!]
		\begin{tabular}{|l|c|}
	\hline
	\textbf{Probability} & \textbf{Value} \\
	\hline
    $P_1$= $\pr{X=0|Y=0}$ & $\dfrac{2}{10}=\dfrac{1}{5}$\\ 
	\hline
	$P_2$= $\pr{X=1|Y=0}$ & $\dfrac{3}{10}$\\ 
	\hline
	$P_3$= $\pr{X=0|Y=1}\times\pr{X=0|Y=0}$ & $\dfrac{1}{9}\times\dfrac{1}{5}=\dfrac{1}{45}$\\ 
	\hline
	$P_4$= $\pr{X=0|Y=1}\times\pr{X=1|Y=0}$ & $\dfrac{2}{9}\times\dfrac{3}{10}=\dfrac{1}{15}$\\ 
	\hline
    $P_5$= $\pr{X=1|Y=1}\times\pr{X=0|Y=0}$ & $\dfrac{3}{9}\times\dfrac{2}{10}=\dfrac{1}{15}$\\ 
	\hline
	$P_6$= $\pr{X=1|Y=1}\times\pr{X=1|Y=0}$ & $\dfrac{2}{9}\times\dfrac{3}{10}=\dfrac{1}{15}$\\ 
	\hline
	$P_7$ & ? \\ 
	\hline
\end{tabular}

		\vspace*{5pt}
		\caption{}
		\label{table:prob2}
	\end{table}
	for the input probabilities.
	The desired probability is then obtained from Table \eqref{table:prob2} as
	
	\begin{align}
		P_7 &=1-(P_3+P_4+P_5+P_6) \\
		&= 1-(\dfrac{1}{45}+ \dfrac{1}{15} + \dfrac{1}{15} + \dfrac{1}{15}) \\
		&= 1-\dfrac{2}{9}\\
		&= \dfrac{7}{9}
	\end{align}
	Hence, the probability that atleast one of the balls drawn from the urn is red is $\dfrac{7}{9}$.
\end{document}
