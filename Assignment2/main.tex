\let\negmedspace\undefined
\let\negthickspace\undefined
\documentclass[journal,11pt,twocolumn]{IEEEtran}
\usepackage{gensymb}
\usepackage{amssymb}
\usepackage[cmex10]{amsmath}
\usepackage{amsthm}
\usepackage[export]{adjustbox}
\usepackage{bm}
\usepackage{longtable}
\usepackage{enumitem}
\usepackage{mathtools}
\usepackage{tikz}
\usepackage[breaklinks=true]{hyperref}
\usepackage{listings}
\usepackage{color}                                            %%
\usepackage{array}                                            %%
\usepackage{longtable}                                        %%
\usepackage{calc}                                             %%
\usepackage{multirow}                                         %%
\usepackage{hhline}                                           %%
\usepackage{ifthen}                                           %%
\usepackage{lscape}     
\usepackage{multicol}
\DeclareMathOperator*{\Res}{Res}

\providecommand{\mbf}{\mathbf}
\providecommand{\pr}[1]{\ensuremath{\Pr\left(#1\right)}}
\renewcommand\thesection{\arabic{section}}
\renewcommand\thesubsection{\thesection.\arabic{subsection}}
\renewcommand\thesubsubsection{\thesubsection.\arabic{subsubsection}}
\renewcommand\thesectiondis{\arabic{section}}
\renewcommand\thesubsectiondis{\thesectiondis.\arabic{subsection}}
\renewcommand\thesubsubsectiondis{\thesubsectiondis.\arabic{subsubsection}}
\hyphenation{op-tical net-works semi-conduc-tor}
\def\inputGnumericTable{}                                 %%
\lstset{
	frame=single, 
	breaklines=true,
	columns=fullflexible
}

\usepackage{graphicx}


\begin{document}
	\title{\huge{Assignment 2}\\AI1110}
	\author{\Large{I Sai Pradeep}\\AI21BTECH11013}
	\maketitle
	\begin{abstract}
		This document contains the solution to problem 1(ix) in 12th ICSE 2019 Paper
	\end{abstract}
	%Question
	\noindent \textbf{Question 1(ix):}
	Two balls are drawn from an urn containing 3 white,5 red and 2 black balls, one by one without replacement. What is the probability that at least one ball is red?\\	
	%Solution
	\textbf{Solution:}
	\begin{enumerate}
		\item Denote the outcome of experiment by a random variable $X_3$ $\in$ \{0,1\}, where $X_3=0$ denote the occurrence that none of the balls drawn is red,and $X_3=1$ denote the occurrence that at least one ball of the balls drawn is red. 
		\item Let the outcome of first draw is denoted by the random variable $X_1$ $\in$ \{0,1\}, where $X_1=0$ denotes the event that first ball drawn is red,and $X_1=1$ denotes the event that first ball drawn is not red.
		\item  Similarly, the outcome of second draw is denoted by the random variable $X_2$ $\in$ \{0,1\}, where $X_2=1$ denotes the event that second ball drawn is red,and $X_2=1$ denotes the event that second ball drawn is not red. 
		See Tables 
		\eqref{table:prob1}
		and 
		\eqref{table:prob2} 
		for the input probabilities.
		\begin{table}[ht!]
			\begin{tabular}{|l|c|}
	\hline
	\textbf{Event} & \textbf{Description} \\
	\hline
	$X = 0$ &  colour of the ball drawn is Black \\
	\hline
	$X = 1$ &  colour of the ball drawn is White \\
	\hline
	$X = 2$ &  colour of the ball drawn is Red \\
	\hline
	$Y = 0$ &  The first draw of the balls \\
	\hline
	$Y = 1$ &  The second draw of the balls \\
	\hline
\end{tabular}

			\vspace*{5pt}
			\caption{}
			\label{table:prob1}
		\end{table}
		\begin{table}[ht!]
			\begin{tabular}{|l|c|}
	\hline
	\textbf{Probability} & \textbf{Value} \\
	\hline
    $P_1$= $\pr{X=0|Y=0}$ & $\dfrac{2}{10}=\dfrac{1}{5}$\\ 
	\hline
	$P_2$= $\pr{X=1|Y=0}$ & $\dfrac{3}{10}$\\ 
	\hline
	$P_3$= $\pr{X=0|Y=1}\times\pr{X=0|Y=0}$ & $\dfrac{1}{9}\times\dfrac{1}{5}=\dfrac{1}{45}$\\ 
	\hline
	$P_4$= $\pr{X=0|Y=1}\times\pr{X=1|Y=0}$ & $\dfrac{2}{9}\times\dfrac{3}{10}=\dfrac{1}{15}$\\ 
	\hline
    $P_5$= $\pr{X=1|Y=1}\times\pr{X=0|Y=0}$ & $\dfrac{3}{9}\times\dfrac{2}{10}=\dfrac{1}{15}$\\ 
	\hline
	$P_6$= $\pr{X=1|Y=1}\times\pr{X=1|Y=0}$ & $\dfrac{2}{9}\times\dfrac{3}{10}=\dfrac{1}{15}$\\ 
	\hline
	$P_7$ & ? \\ 
	\hline
\end{tabular}

			\vspace*{5pt}
			\caption{}
			\label{table:prob2}
		\end{table}
		The desired probability is then obtained from
		\eqref{eq:prob_1}
		\begin{align}
		\pr{X_3=1}&= 1-\pr{X_1=1} \times \pr{X_2=1/X_1 =1}
	    \label{eq:prob_1}
		\end{align}
		\begin{align}
			\implies \pr{X_3 = 1} &= 1-{{\dfrac{1}{2}} \times \dfrac{4}{9}}
			\\
			&= \dfrac{7}{9}
	    \end{align}
		$\therefore$ The probability of drawing at least one red ball is \dfrac{7}{9}
		
	\end{enumerate}
\end{document}